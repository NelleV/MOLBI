\documentclass{article}
%\usepackage[latin1]{inputenc}
\usepackage{graphicx,amssymb,amsmath,amsbsy,MnSymbol} % extensions pour maths
\usepackage{graphicx,mathenv}           % extensions pour figures
\usepackage[T1]{fontenc}        % pour les charactères accentués 
\usepackage[utf8]{inputenc} 
\usepackage{multicol}
\usepackage{wrapfig}
\usepackage{stmaryrd} % Pour les crochets d'ensemble d'entier
\usepackage{float}  % Pour placer les images là ou JE veux.

\DeclareMathOperator{\tr}{tr}
\DeclareMathOperator{\argmax}{argmax}
\DeclareMathOperator{\argmin}{argmin}
\DeclareMathOperator{\cov}{cov}


\setlength{\parindent}{0.0in}
\setlength{\parskip}{0.1in}
\setlength{\topmargin}{-0.4in}
\setlength{\topskip}{0.7in}    % between header and text
\setlength{\textheight}{9in} % height of main text
\setlength{\textwidth}{6in}    % width of text
\setlength{\oddsidemargin}{0in} % odd page left margin
\setlength{\evensidemargin}{0in} % even page left margin
%
%% Quelques raccourcis clavier :
\def\slantfrac#1#2{\kern.1em^{#1}\kern-.3em/\kern-.1em_{#2}}
\def\b#1{\mathbf{#1}}
\def\bs#1{\boldsymbol{#1}}
\def\m#1{\mathrm{#1}}
\bibliographystyle{acm}
%
\newcommand{\greeksym}[1]{{\usefont{U}{psy}{m}{n}#1}}
\newcommand{\inc}{\mbox{\small\greeksym{d}\hskip 0.05ex}}%
\pagenumbering{arabic}
\date{\today}
\title{HW1 - Detecting Differentially Expressed Genes}
\author{Nelle Varoquaux}
\begin{document}
\maketitle

\section{Model}
Given the phenotype, all expressions over genes and samples are independant.
We assume that the first $d_1$ genes are distributed with
$\mathcal{B}(p_1^i)$, given $Y = i$, and the next genes are
$\mathcal{B}(p_2^i)$, \dots

\begin{align*}
P(X | Y = i) & = & \prod_{j = 1}^d p_j^i (1 - p_j)^{1 - i} \\
	     & = & p_1^{id_2}p_1^{(1 - i)d_2} p_2^{id_2}p_2^{(1 - i)d_2}
	     \dots p_k^{id_1}p_k^{(1 - i)d_k}
\end{align*}

\section{The Data}

The genes from $d_2$ and $d_3$ are differentially expressed. The set of genes
from $d_2$ are harder to distinguish.

The more samples we have, the easier it will be to differentiate the harder
set.

\section{Mutual Information Estimator}

For empirical data, we have:
\begin{align*}
\hat{I}(X, Y) & = & H(X) - H(X | Y) \\
	      & = & - \hat{p}(X = 0) \log(\hat{p}(X = 0)) - \hat{p}(X = 1)
	      \log(\hat{p}(X= 1)) \\
	      &  & + p(Y = 0)\{ \hat{p}(X = 0 | Y = 0) \log(\hat{p}(X= 0 | Y = 0)) +
		   \hat{p}(X = 0 | Y = 1) \log(\hat{p}(X = 0 | Y = 1)) \} \\
	      &  & + p(Y = 1) \{\hat{p}(X = 1 | Y = 0) \log(\hat{p}(X = 1 | Y = 0)) +
		   \hat{p}(X = 1 | Y = 1) \log(\hat{p}(X = 1 | Y = 1)) \} \\
	      & = & - \frac{n_0}{n} \log(\frac{n_0}{n}) - \frac{1 - n_0}{n}
	      \log(\frac{1 - n_0}{n}) +  \\
	      &  & \frac{1}{2} \{\frac{n_{00}}{\frac{1}{2} n} \log \( 
	      \frac{n_{00}}{\frac{1}{2} n} \) +  \frac{1 - n_{00}}{\frac{1}{2} n} \log \(
	      \frac{1 - n_{00}}{\frac{1}{2} n} \) + \frac{n_{01}}{\frac{1}{2} n} \log \( 
	      \frac{n_{01}}{\frac{1}{2} n} \) +  \frac{1 - n_{01}}{\frac{1}{2} n} \log \(
	      \frac{1 - n_{01}}{\frac{1}{2} n}\) \}
\end{align*}

We can also calculate the mutual information estimator analytically:

% TODO

\section{Mutual Information and Different Sample Sizes}

\begin{figure}
\begin{center}
\includegraphics[width=300px]{./images/MI.png}
\end{center}
\caption{Mutual information for $n=10$, $n=100$ \& $n=1000$}
\end{figure}

For $n = 10$, it is hard to distinguish the two different distributions. Yet,
for the last distribution, ie the last 100 genes, the mutual information is
much higher than for the first two. With $n$ increasing, we start to
distinguish the second set of genes, corresponding to the second distribution.
The mutual information for the first set of genes decreases, and is almost 0
for $n = 1000$, which means that the two distributions for $Y = 0$ and $Y = 1$
are the same.
The mutual information can therefore be used to detect DE genes, by setting a
threshold.


\section{Discovering DE genes using Mutual Information and ROC Curve}
\begin{figure}
\begin{center}
\includegraphics[width=300px]{./images/MI_ROC.png}
\end{center}
\caption{ROC curve for setting a threshold on the Mutual Information}
\end{figure}

One can use a threshold $\tau$ on the mutual information to infer whether a
gene is DE or not. To choose such a threshold, we plot the ROC curve.

We can observe that when the number of samples increases, the algorithm yields
better results: the ROC curve approaches the optimal ROC curve for $n = 1000$.

Choosing the threshold $\tau$ implies making a compromise between the ratio of
true positive and of false positive. Ideally, we would like to have 100\% of
true positives, and no false positives.

In the ROC space, the Identity function ($f(y) = x$) corresponds to the ROC
curve of a random decision function. In order to find the best $\tau$, we need
to find $\tau$ for which the distance of the point of the ROC curve
corresponding to $\tau$ is maximum.

For $n=10$, $\tau_{opt} = 0.076246$, which yields $200$ True Positives, $878$
False Positive, $122$ True Negatives, and $0$ False Positives.

For $n=100$, $\tau_{opt} = 0.016230$, which yields $200$ True Positives, $970$
False Positive, $30$ True Negatives, and $0$ False Positives.

For $n = 1000$, $\tau_{opt} = 0.002162$, which yields $183$ True Positives, $43$
False Positives, $957$ True Negatives, $17$ False Negative.


\section{DE genes and Fisher's Exact Test}
\begin{figure}
\begin{center}
\includegraphics[width=300px]{./images/FET_ROC.png}
\end{center}
\caption{ROC curve for Fisher's Exact Test}
\end{figure}

In order to plot the ROC curve, we threshold the distance of the pvalue to
$\frac{1}{2}$.

For $n=10$ $\tau_{opt} = 0.417$ which yields $194$ True Positives, $165$
False Positives, $835$ True Negatives, $6$ False Negative.

For $n=100$ $\tau_{opt} = 0.476$ which yields $183$ True Positives, $46$
False Positives, $954$ True Negatives, $17$ False Negative.

For $n=1000$ $\tau_{opt} = 0.475$ which yields $192$ True Positives, $51$
False Positives, $949$ True Negatives, $8$ False Negative.

\section{Discussion}
With few samples, the mutual informations performs horribly, while the Fisher
Exact Test performs quite well.

With many samples ($n=1000$), the mutual information yields better results
than the Fisher Exact Test.
\end{document}
